% Options for packages loaded elsewhere
\PassOptionsToPackage{unicode}{hyperref}
\PassOptionsToPackage{hyphens}{url}
%
\documentclass[
  11pt,
]{article}
\title{Adding species to stdpopsim}
\author{true \and true \and true}
\date{}

\usepackage{amsmath,amssymb}
\usepackage{lmodern}
\usepackage{iftex}
\ifPDFTeX
  \usepackage[T1]{fontenc}
  \usepackage[utf8]{inputenc}
  \usepackage{textcomp} % provide euro and other symbols
\else % if luatex or xetex
  \usepackage{unicode-math}
  \defaultfontfeatures{Scale=MatchLowercase}
  \defaultfontfeatures[\rmfamily]{Ligatures=TeX,Scale=1}
\fi
% Use upquote if available, for straight quotes in verbatim environments
\IfFileExists{upquote.sty}{\usepackage{upquote}}{}
\IfFileExists{microtype.sty}{% use microtype if available
  \usepackage[]{microtype}
  \UseMicrotypeSet[protrusion]{basicmath} % disable protrusion for tt fonts
}{}
\makeatletter
\@ifundefined{KOMAClassName}{% if non-KOMA class
  \IfFileExists{parskip.sty}{%
    \usepackage{parskip}
  }{% else
    \setlength{\parindent}{0pt}
    \setlength{\parskip}{6pt plus 2pt minus 1pt}}
}{% if KOMA class
  \KOMAoptions{parskip=half}}
\makeatother
\usepackage{xcolor}
\IfFileExists{xurl.sty}{\usepackage{xurl}}{} % add URL line breaks if available
\IfFileExists{bookmark.sty}{\usepackage{bookmark}}{\usepackage{hyperref}}
\hypersetup{
  pdftitle={Adding species to stdpopsim},
  hidelinks,
  pdfcreator={LaTeX via pandoc}}
\urlstyle{same} % disable monospaced font for URLs
\usepackage[margin=1in]{geometry}
\usepackage{graphicx}
\makeatletter
\def\maxwidth{\ifdim\Gin@nat@width>\linewidth\linewidth\else\Gin@nat@width\fi}
\def\maxheight{\ifdim\Gin@nat@height>\textheight\textheight\else\Gin@nat@height\fi}
\makeatother
% Scale images if necessary, so that they will not overflow the page
% margins by default, and it is still possible to overwrite the defaults
% using explicit options in \includegraphics[width, height, ...]{}
\setkeys{Gin}{width=\maxwidth,height=\maxheight,keepaspectratio}
% Set default figure placement to htbp
\makeatletter
\def\fps@figure{htbp}
\makeatother
\setlength{\emergencystretch}{3em} % prevent overfull lines
\providecommand{\tightlist}{%
  \setlength{\itemsep}{0pt}\setlength{\parskip}{0pt}}
\setcounter{secnumdepth}{5}
\ifLuaTeX
  \usepackage{selnolig}  % disable illegal ligatures
\fi

\begin{document}
\maketitle

\hypertarget{outline}{%
\section{Outline}\label{outline}}

\begin{itemize}
\tightlist
\item
  recap goals and structure of stdpopsim

  \begin{itemize}
  \tightlist
  \item
    two sets of goals:

    \begin{itemize}
    \tightlist
    \item
      methods development/testing
    \item
      inference (this is perhaps the major goal this paper is
      addressing?)
    \item
      arguably adding species opens stdpopsim to the community whose
      primary concern is inference rather than methods development
    \end{itemize}
  \end{itemize}
\item
  discussion of resources necessary to simulate/add species

  \begin{itemize}
  \tightlist
  \item
    recombination map
  \item
    annotation
  \item
    assembly

    \begin{itemize}
    \tightlist
    \item
      not just contigs
    \end{itemize}
  \item
    demographic model?

    \begin{itemize}
    \tightlist
    \item
      at least reasonable Ne
    \end{itemize}
  \end{itemize}
\item
  discussion of types of species in terms of:

  \begin{itemize}
  \tightlist
  \item
    available resources
  \item
    what goal adding the species would address
  \end{itemize}
\item
  description of how to add a species to the catalog and/or write a yaml
  for a new species

  \begin{itemize}
  \tightlist
  \item
    this will depend on how we're ultimately moving forward - using the
    current python-based method, or switching to a yaml-based method
  \end{itemize}
\end{itemize}

\begin{enumerate}
\def\labelenumi{\arabic{enumi}.}
\tightlist
\item
  Overall goals and structure of stdpopsim A. stdpopsim is a
  community-maintained resource intended to provide easy access to
  simulation frameworks

  \begin{enumerate}
  \def\labelenumii{\roman{enumii}.}
  \tightlist
  \item
    coding population genetic simulation models can be arduous and
    error-prone, so stdpopsim provides a way to:
  \end{enumerate}

  \begin{itemize}
  \tightlist
  \item
    avoid re-making common models
  \item
    provide standard benchmarks for methods development and testing
  \end{itemize}

  \begin{enumerate}
  \def\labelenumii{\roman{enumii}.}
  \setcounter{enumii}{1}
  \tightlist
  \item
    and a resource for empirical researchers, eg. power analyses or
    sanity checks (Adrion et al.~2020) B. feedback from 2020/2021
    workshops made it clear that the prospective community included
    empiricists who are especially concerned with using stdpopsim for
    inference for individual species that mostly not already in
    stdpopsim catalog
  \end{enumerate}
\item
  Community-based expansion of the number and variety of species and
  their demographic scenarios included in the stdpopsim catalog is
  useful for both of these major goals shared by the population genetics
  community, methods development and inference A. when first published,
  the stdpopsim catalog included 6 species: \emph{Homo sapiens},
  \emph{Pongo abelii}, \emph{Canis familiaris}, \emph{Drosophila
  melanogaster}, \emph{Arabidopsis thaliana}, and \emph{Escherichia
  coli.}

  \begin{itemize}
  \tightlist
  \item
    when did additional demographic scenarios start getting added? B. in
    April 2021 we held a ``Growing the Zoo'' hackathon to involve the
    community in adding additional species and demographic scenarios of
    interest
  \item
    the catalog now includes an additional \textbf{X} species (how many?
    the catalog doc still only includes the original 6?), as well as
    multiple demographic scenarios for \emph{Homo sapiens}, \emph{Pongo
    abelii}, \emph{Drosophila melanogaster}, and \emph{Arabidopsis
    thaliana.} (others?)
  \item
    however there were some species that were not ideal for inclusion in
    the catalog, because they don't yet have the necessary genomic
    resources C. stdpopsim has become a popular resource and there is
    clearly a desire/need for adding additional species for both goals
    of methods development and empirical inference
  \item
    choosing species/demographic scenarios to add is potentially
    challenging
  \item
    for methods development, they have to be appropriate for the goals
    of the method
  \item
    for empirical studies, the species is obvious, but whether it is
    appopriate for inclusion in the catalog or even to use stdpopsim may
    be less clear
  \end{itemize}
\item
  Therefore this paper is intended as a resource for both methods
  developers and empirical researchers to choose and add appropriate
  species (and demographic scenarios) to the stdpopsim catalog A. builds
  off the ``Growing the Zoo'' hackathon, held along side probgen in
  April 2021

  \begin{enumerate}
  \def\labelenumii{\arabic{enumii}.}
  \tightlist
  \item
    what we learned from the hackathon (and workshops)?
  \end{enumerate}
\end{enumerate}

\hypertarget{abstract}{%
\section{Abstract}\label{abstract}}

\hypertarget{introduction}{%
\section{Introduction}\label{introduction}}

\hypertarget{methods}{%
\section{Methods}\label{methods}}

\hypertarget{results}{%
\section{Results}\label{results}}

\hypertarget{discussion}{%
\section{Discussion}\label{discussion}}

\hypertarget{references}{%
\section{References}\label{references}}

\end{document}
